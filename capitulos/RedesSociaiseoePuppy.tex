\chapter{Redes Sociais e o ePuppy}
Para a realização do projeto, foi fundamental uma divisão em duas etapas. O primeiro momento, procura-se reter nas especificidades de uma rede social, aprimorando-se em suas características básicas e em seu comportamento. Após esse momento, foram realizadas pesquisas sobre outros sistemas parecidos e em quais ambientes o ePuppy poderia se expandir, para não ser apenas uma rede social, um sistema de clínica veterinária ou até mesmo um sistema de segurança para os animais, mas sim, um sistema que engloba todas essas caraterísticas, tornando-se um projeto mais completo e inovador.

\section{Redes Sociais}
A rede mundial, ou {\it Internet} surge em plena guerra fria, com objetivos militares específicos e um importante meio de comunicação acadêmica. Em meados da década de 90 a internet começou a alcançar a população em geral, a com isso surgiu também a utilização de interface gráfica e a criação de sites mais dinâmicos, visualmente mais interessantes.
\\
\indent
A partir dos anos 2000 a conexão entre as pessoas ficou ainda mais fácil e com o aprimoramento desta tecnologia da informação, outro tipo de serviço de comunicação começou a ganhar forças: as {\bf Redes Sociais}. Estrutura social composta por pessoas e organizações, conectadas por vários tipos de relações, que compartilham valores e objetivos comuns, sejam eles, imagens, informações, vídeos e arquivos de áudio.
\\
\indent
Desde 2006, esses serviços viraram “febre” no mundo inteiro chegando a haver uma grande descriminação para aqueles que não usavam.
\\
\indent
Mas como é possível um convívio social à grandes distâncias? Segundo Brown e Barnett, “o agrupamento de indivíduos, de acordo com as posições que resultam dos padrões essenciais de relações de obrigação, constitui a estrutura social de uma sociedade”. Portanto, esses valores unidos aos fatores virtuais, fazem com que haja um estrapolamento das quatro paredes convencionais, tornando assim um convívio social a grandes distâncias.
\\
\indent
Hoje, as principais redes sociais sejam elas profissionais, de relacionamentos, politicas ou comunitárias são: LinkedIn, Myspace, Facebook, Twitter, entre outros.

\section{ePuppy}
O ePuppy surge como um projeto designado para quem procura mais praticidade em questão de segurança, informação e interação. Para que isso fosse precisamente implementado no sistema subdividimos os três quesitos em vários casos de uso.
\\
\indent
O sistema funcionará com basicamente três tipos de usuários (Proprietários, clínicas e veterinários). E disponibilizando as mais variadas tarefas para esses usuários, passando de controle e segurança de animais dos proprietários, até ao controle de proprietários e veterinários em clínicas veterinárias.
\\
\indent
Os proprietários do sistema poderão cadastrar a maior quantidade de animais possíveis, adicionando nomes, espécie, raça dentre outros dados. Cada animal será identificado por um {\it QR Code} diferenciado, na qual, a priori, serão adicionadas em coleiras, dependendo do animal. O {\it QR Code} terá dados acerca do dono e do pet, garantindo assim, que o proprietário seja facilmente notificado, por quem achar o animal (caso este esteja perdido). Os {\it pets} ainda terão um histórico hospitalar, para que os veterinários permitidos pelo dono, tenham todo um apanhado detalhado de suas consultas passadas, obtendo assim: Remédios, consultas, cirurgias e doenças acerca do animal.
\\
\indent
Por o ePuppy ser um sistema de segurança e também, uma rede social, os usuários em geral poderão compartilhar publicações como: informações e arquivos em geral, comentar publicações, ter amigos, grupos, ver perfis de outros usuários e seus animais. Além de todos esses atributos, o sistema ainda disponibilizará aos usuários, {\it login} por redes sociais, fazendo com que publicações e ações em relação aos seus animais, clínicas e veterinários, sejam publicadas também nessas outras redes, caso o usuário atribua permissão para tal procedimento.
\\
\indent
Já do lado das clínicas veterinárias, o ePuppy procura proporcionar meios para que essas divulguem os seus serviços, na própria rede social e em outras, além de possibilitar um maior controle de seus pacientes e veterinários, e um maior ponto de acesso de uma maior quantidade de usuários.
\\
\indent
A rede social ainda possibilita um auxílio ao trabalho dos veterinários, principalmente porque esses terão um maior acesso às antigas consultas de seus pacientes (animais), podendo ser mais eficientes durante o receitamento de determinado remédio, consulta, cirurgia, além de um maior monitoramento desses.

\section{Trabalhos Relacionados}
Para o desenvolvimento do ePuppy, tomou-se como base um sistema que já está em funcionamento, chamado Meu Peludo. Nele, o proprietário compra a coleira do sistema, que contém um {\it QR Code} agregada a um {\it link} da {\it internet}. Nela estão contidos os dados do cachorro, do dono e um formulário, para o preenchimento da pessoa que encontrar o animal (caso este esteja perdido), fazendo com que este consiga entrar em contato com o dono. Nos apoiamos nessa funcionalidade em especial, e fomos aderindo novas, buscando deixar o ePuppy mais completo e mais amplo.
\\
\indent
Durante o desenvolvimento da mais nova rede social, conhecemos um sistema chamado TagPet, que tem muitas das funcionalidades que estávamos desenvolvendo. Porém, a ideia do nosso sistema é ser uma rede social para proprietários de animais domésticos, e estamos desenvolvendo um requisito para que as publicações feitas no ePuppy, possam ser disparadas nas demais redes sociais do usuário, caso ele tenha se cadastrado utilizando os dados destas. Além disso, as pessoas vão poder compartilhar em quais clínicas ou {\it pet shops} favoritos estão, funcionando como uma publicidade para esses lugares.



