\chapter{Considerações Finais}

Portanto, o presente trabalho apresenta o ePuppy, um sistema que tem por objetivo ajudar os proprietários numa maior segurança em relação aos seus animais, informação por meio de notícias compartilhadas por outras pessoas e uma interação com clínicas e veterinários desejados.
\\
\indent
O sistema foi desenvolvido para web, que permite aos demais usuários acessá-lo a qualquer momento lançando mão de algum dispositivo com acesso à internet.

\section{Trabalhos Futuros}
O ePuppy continua em desenvolvimento com duas partes do projeto, a {\it front-end} e {\it back-end}. Cada parte envolvida está encarregada de dar prosseguimento ao andamento do sistema. O {\it Front-end} cuida da parte gráfica, enquanto o {\it Back-end} cuida do controle e da lógica por trás do software. Por enquanto o projeto ainda não se encontra hospedado, por se encontrar em fase de testes lógicos.
\\
\indent
Como trabalho futuro, pretendemos desenvolver e implementar as demais funcionalidades que estão em andamento e outras como: a utilização do {\it QRCode} para vincular ao animal, ACL (lista de controle de acessos), {\it upload} de diversos arquivos (imagens dos usuários e animais e anexos do veterinário durante consulta), ePuppy {\it Mobile} (versão do ePuppy para dispositivos móveis), histórico hospitalar, incluir páginas e acesso para clínicas e veterinários e adicionar espaço para vendas de produtos e serviços de {\it petshops}, além de expandir o sistema de modo a se tornar uma grande rede social.

