\chapter{Referencial Teórico}
Para o desenvolvimento do ePuppy foi necessário seguir alguns critérios da Engenharia de Software seguindo alguns pensadores da Área, para facilitar a aplicação e a elaboração de um sistema consistente.
\\
\indent
Todo o desenvolvimento de um software é “sistemático, disciplinado e quantificável”, ou seja, precisamos de disciplina, mas também adaptabilidade e agilidade. (PRESSMAN, 2006).
\\
\indent
Para a produção de um software de alta qualidade, engenheiros de software procuram a maneira mais eficaz, adotando uma abordagem sistemática e organizada em seu trabalho. (SOMMERVILLE, 2011).
\\
\indent
No entanto, a engenharia de software seleciona métodos mais apropriados para um conjunto de circunstâncias e uma abordagem mais criativa e menos formal, sendo fortemente eficaz. (SOMMERVILLE, 2011).
\\
\indent
Dessa forma, a engenharia de software utiliza uma tecnologia de camadas, apoiando-se num compromisso organizacional com a qualidade. A partir disso, tem-se o alicerce dessa camada, denominada camada de processo que define a metodologia que devem ser estabelecidos para a efetiva utilização da tecnologia da engenharia de software. Após o processo, tem-se os métodos que fornecem a técnica de “como fazer” para a construção de softwares e por fim, a ferramentas que fornecem apoio automatizado ou semi-automatizado para o processo e para os métodos. (PRESSMAN, 2006).
\\
\indent
Apoiando-se ainda na qualidade do projeto, é necessário elaborar uma proposta (obter um contrato para elaborar proposta), planejar e desenvolver um cronograma do projeto, avaliar o custo do projeto, monitorar e revisar, selecionar e avaliar o pessoal e elaborar relatórios de apresentação. (SOMMERVILLE, 2011).
\\
\indent
Para que todos esses passos abordados anteriormente sejam efetuados de forma correta, consistente e acima de tudo, com clareza, é necessário abstrair com clareza os requisitos de um sistema, ou seja, o problema do cliente. Os requisitos de um sistema são descrições dos serviços fornecidos pelo sistema e as suas restrições operacionais, ou seja, refletem a necessidade do cliente. (SOMMERVILLE, 2011).
\\
\indent
Segundo Pressman (PRESSMAN, 2006) Entender os requisitos de um problema está entre as tarefas mais difíceis enfrentadas por um engenheiro de software, porque surpreendentemente, na maioria dos casos, o cliente não sabe o que é necessário, e não tem um bom entendimento das características e funções que vão oferecer benefícios.
