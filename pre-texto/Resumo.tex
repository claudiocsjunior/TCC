\begin{resumo}
\noindent
A adoção de animais domésticos nas famílias vem crescendo notoriamente no Brasil e em parte do mundo. Cada vez mais, esses animais estão sendo humanizados e passam a compartilhar de alguns dos mesmos privilégios de pessoas comuns, como poder circular em lugares públicos, ter acesso à saúde e tratamento médico especializado, dispondo de uma rede de lojas dedicadas a oferecer uma variedade de produtos e serviços para esse novo mercado de consumo. Diante da nova mudança de comportamento, e visando um mercado futuro promissor, o ePuppy é um projeto de pesquisa e inovação que é destinado à criação de uma rede social de animais de estimação. A ideia possibilita reunir em um único espaço informações acerca dos animais, veterinários, fornecedores de produtos, clínicas e hospitais. O projeto interligará todos os usuários do sistema, criando uma grande rede interna em conjunto com  as mais variadas redes sociais. O sistema também reunirá formas de ajudar seus clientes a acharem seus animais de estimação, caso esses se encontrem perdidos ou desaparecidos, por meio de coleiras personalizadas com QR Code. 

\noindent
\textbf{Palavras-chave}: Rede Social; Animais; QR Code; Localização.
\end{resumo}
